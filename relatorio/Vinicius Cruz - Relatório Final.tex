\documentclass[12pt, a4paper]{article}
\usepackage[utf8]{inputenc}
\usepackage[brazilian]{babel}
\usepackage{amsmath}
\usepackage{amsfonts}
\usepackage{amssymb}
\usepackage{verbatim}
\usepackage[T1]{fontenc}
\usepackage{graphicx}
\usepackage{physics}
\usepackage{siunitx}
\usepackage[top=2cm, bottom=2cm, left=3cm, right=2cm]{geometry}
\usepackage{makeidx}
\input{commands}
\usepackage[tocflat]{tocstyle}
\usetocstyle{standard}



\makeatletter
\renewcommand\tableofcontents{
  \null\hfill\textbf{\Large\contentsname}\hfill\null\par
  \@mkboth{\MakeUppercase\contentsname}{\MakeUppercase\contentsname}
  \@starttoc{toc}
}
\makeatother

\author{COPES - UFS}
\title{Modelo_Relatorio_Parcial}
\date{}
%\Marginsize marginsize { 1 centímetro } {2 } { 4 em em} { 6 pt } 

\begin{document}

\begin{figure}[!h]
    \centering
    \includegraphics[scale=1.2]{ufs.png}

  \end{figure}
%%--CABEÇALHO--%%

 \begin{center}
 
 \Large UNIVERSIDADE FEDERAL DE SERGIPE
PRÓ-REITORIA DE PÓS-GRADUAÇÃO E PESQUISA
COORDENAÇÃO DE PESQUISA
\vspace{10mm} 

\normalsize PROGRAMA DE INICIAÇÃO CIENTÍFICA VOLUNTÁRIA – PICVOL

\vspace{20mm}

\textbf{\Large CONTROLE ÓPTIMO EM TERAPIAS DE CÂNCER}

\vspace{20mm}
 {\normalsize Área do conhecimento: Matemática\\
Subárea do conhecimento: Equações Diferenciais Parciais\\
Especialidade do conhecimento: Oncologia Matemática\\}
\vspace{20mm}

{\normalsize Relatório Final\\
Período da bolsa: de 08/2017 a 08/2018}
\vspace{10mm}

 {\large Este projeto é desenvolvido de forma voluntária}
\vspace{10mm}

\large{PICVOL}


 \end{center}

 
\newpage

 
 \begin{flushleft}
 
\tableofcontents 


\end{flushleft}
\newpage

\section{ Introdução}

Câncer é o conjunto de mais de 100 doenças, cujo ponto em comum é o crescimento desordenado de células que invadem tecidos e órgãos.\

Nos últimos anos o número de casos tem aumentado e segundo estimativas da Organização Mundial de Saúde o número de novos casos só irão aumentar nos próximos 20 anos. Tornando o câncer a principal causa de morte, superando as doenças cardiovasculares e cerebrovasculares. \

Um dos métodos utilizados para combater o câncer é a radioterapia. Consiste em aplicar feixes de radiações ionizantes, com o objetivo de, através de uma alteração no DNA, impedir a divisão das  tumorais.\

Com o intuito de estudar a dinâmica do crescimento de tumores, junto com o tratamento de radioterapia, fomos introduzidos a temas como o Cálculo Variacional e a Teoria de Controle. Além da introdução de conteúdos novos, foi possível ver na prática, aplicações de conteúdos apresentados em disciplinas da graduação, como o Cálculo e Álgebra Linear. Aumentando assim a compreensão e o interesse por essas disciplinas.\

Continua...

\newpage

\section{Objetivos}
Introduzir o tema da dinâmica do crescimento de tumores para que fosse possível para o aluno entender as publicações atuais, na área. Para que seja possível dar os primeiros passos na pesquisa em direção a publicações próprias.

\newpage

\section{Metodologia}

A pesquisa é classificada, tanto do ponto de vista dos procedimentos metodológicos, quanto da natureza, como uma pesquisa bibliográfica, uma vez que os estudos foram feitos a partir de materiais já publicados. Periodicamente, houve encontros entre orientando e orientador para discussão e exposição do conteúdo estudado. Ao final, para colocar em prática os conteúdos apresentados, foi proposta a implementação de código computacional para reproduzir as simulações obtidas em um dos trabalhos, usados como fonte de estudo.
\cite{Burns2014}
\newpage

\section{Resultados e discussões }
\section{Conclusões}
\section{Perspectivas}

\newpage

\section{Referências bibliográficas}
\bibliography{pibic-2018.bib}
	
\newpage

\section{Outras atividades}
Além dos encontros em sala para apresentação do conteúdo introdutório ao assunto,
participei, como atividade complementar, do 27º Encontro de Iniciação Científica da UFS.
Minicurso sobre GERENCIAMENTO DE REFERÊNCIAS BIBLIOGRÁFICAS PARA
TRABALHOS DE PESQUISA E ARTIGOS CIENTÍFICOS. No dia 21/11/2017. E do II SSMAC 2018 (Série de Seminários de Matemática Aplicada e Computacional). Palestra sobre UM MODELO PARA CONSTRUIR FOTOGRAFIAS PANORÂMICAS. Dia 20/04/2018.\

\end{document}
