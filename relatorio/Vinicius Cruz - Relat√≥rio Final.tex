%% Template para dissertação/tese na classe UFPEthesis
%% versão 0.9.2
%% (c) 2005 Paulo G. S. Fonseca
%% www.cin.ufpe.br/~paguso/ufpethesis

\documentclass{ufpethesis}

%% Preâmbulo:
%% coloque aqui o seu preâmbulo LaTeX, i.e., declaração de pacotes,
%% (re)definições de macros, medidas, etc.

%% Identificação:

\university{Universidade Federal de Sergipe}

\address{Este projeto é desenvolvido de forma voluntária - PICVOL}

\institute{Pró-Reitoria de Pós-Graduação e Pesquisa}

\department{Coordenação de Pesquisa}

\program{rograma Institucional de Bolsas de Iniciação Científica(PIBIC) - COPES/UFS\\
Relatório Parcial\\
Período: 08/2015 a 01/2016}

\majorfield{Matemática}

\title{Controle Ótimo em Terapias de Câncer}

\date{}

\author{Bolsista: Vinícius dos Santos Cruz\\
Matrícula: }

\adviser{Paulo de Souza Rabelo}

\coadviser{Gastão}


\begin{document}

%% Parte pré-textual
\frontmatter

% Folha de rosto
\frontpage

\resumo
esse é um resumo.
\begin{keywords}
Espaços vetoriais; operadores lineares; teorema espectral, integrais sobre superfícies.
\end{keywords}

% Sumário
\tableofcontents

% Parte Textual
\mainmatter

\chapter{Introdução}

As pesquisas em Matemática nas \'{u}ltimas d\'{e}cadas t\^{e}m vivido um enorme desenvolvimento devido a sua extensa aplicação nas v\'{a}rias \'{a}reas da ci\^{e}ncia. Destacamos na matem\'{a}tica as Equações Diferenciais Parciais (EDP) por sua grande influência em modelos que surgem da física e da biologia, a exemplo das Equações de Reação e Difusão, dos Sistemas de Leis de Conservação, da Hidrodinâmica, da formação de tecidos e padrões dentro de organismos vivos e crescimento de tumores. Uma outra classe importante de EDPs surge de problemas em geometria diferencial, com grande destaque para o problema de Yamabe.

Neste projeto, pretendemos estudar o operador de Laplace - Beltrami sobre uma variedade conexa compacta sem bordo, que é um dos temas daquilo que podemos chamar de "Análise Geométrica", em sua versão mais básica. Para desenvolver o estudo sobre o tema precisamos de muitas ferramentas básicas da matemática, como os cursos de cálculo da graduação e outros pré-requisitos necessários ao estudo.

A motivação para a escolha deste tema, que visa o aprofundamento de conhecimento e a possibilidae de elaboração de algo novo, deve-se ao fato de apenas os conhecimentos adquiridos nos cursos da grade curricular da graduação, não tornar o aluno capaz de entrar em contato, de modo natural, com conceitos um pouco mais sofisticados do ponto de vista matemático, inerentes às áreas teóricas e aplicadas.

Pretendemos estudar e assim elaborar um texto sobre o tema com uma abordagem matemática simples e que sirva de referência aos alunos que pretendam estudar o assunto.


\end{document}